\section{Choice Calculus}
\label{bk-cc}
The choice calculus~\cite{Walk13thesis, Erwig:2011:CCR:2063239.2063245} is a
formal metalanguage for the static, explicit representation of
\textit{variation} of an object language. For example, in predicate logic, given
two expressions \[ex_1\] and \[ex_2\]:

\[ex_1 = b \Leftrightarrow c \wedge (d \vee e) \wedge \neg c \]

\[ex_2 = a \wedge \wedge (d \vee e) \wedge \neg c \]

We can collapse both expressions to a single example that makes the variation
distinguishing the two explicit via a \textit{choice} :

\[ex_{\chc[A]{1,2}} = \chc[A]{b \Leftrightarrow c, a} d \wedge \neg c  \]

A Choice is a ternary relationship between a choice variable, called a
\textit{dimension}, and two alternatives. In $ex_{\chc{A}[1,2]}$ the choice is
indexed by dimension \textit{A}, with the left alternative being $ b
\Leftrightarrow c$ and its right alternative being $a$. Any expression that
includes a choice is called \textit{variational}, while we call any expression
that does not include a choice \textit{plain}, such as $ex_1$ and $ex_2$. A
plain expression can be derived from a variational one via the process of
\textit{selection} and is called a \textit{variant}. In order to select a
variant, one needs to know which variant to select. This is done via a
\textit{configuration} which is a mapping from dimensions to alternatives. To
select the $i$th alternative, for all dimension choices indexed by $d$, in
expression $e$, we write \tsel{e}. For example to select $ex_1$ from
$ex_{\chc[A]{1,2}}$ we would write \tsel[A.1]{ex_{\chc[A]{1,2}}}.
