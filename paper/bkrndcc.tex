\subsection{Choice Calculus}
\label{bk-cc}
%%% Intro
This section is a brief introduction to the choice calculus~\cite{Walk13thesis,
  Erwig:2011:CCR:2063239.2063245}. It is meant as a brief overview meant to
equip the reader with enough concepts of the calculus to understand the
rest of the paper. For a more exhaustive treatment see


%% Syntax
The choice calculus is a
formal metalanguage for the static, explicit representation of
\textit{variation} of an object language. For example, in predicate logic, given
two expressions $ex_1$ and $ex_2$:

\[ex_1 = b \Leftrightarrow c \wedge (d \vee e) \wedge \neg c \]

\[ex_2 = a \wedge (d \vee e) \wedge \neg c \]

We can collapse both expressions to a single example that makes the variation
distinguishing the two explicit via a \textit{choice} :
\[ex_{\chc[A]{1,2}} = \chc[A]{b \Leftrightarrow c, a} \wedge (d \vee e) \wedge
  \neg c \]
A Choice is a ternary relationship between a choice variable, called a
\textit{dimension}, and two alternatives. In $ex_{\chc[A]{1,2}}$ the choice is
indexed by dimension \textit{A}, with the left alternative being $ b
\Leftrightarrow c$ and its right alternative being $a$. Any expression that
includes a choice is called \textit{variational}, while we call any expression
that does not include a choice \textit{plain}, such as $ex_1$ and $ex_2$.

%% Selection and synchronization
A plain expression can be derived from a variational one via the process of
\textit{selection} and is called a \textit{variant}. In order to select a
variant, one needs to know which variant to select. This is done via a
\textit{configuration} which is a mapping from \change{in code this is a mapping
  from dimensions to booleans? Is this the right language?}dimensions to
alternatives. To select the $i$th alternative, for all dimension choices indexed
by $d$, in expression $e$, we write \tsel{e}. In the general case there are
allowed to be $i$ alternatives, for the rest of this paper we will restrict the
notation to only two. For example to select $ex_2$ from $ex_{\chc[A]{1,2}}$ we
would write \tsel[A.2]{ex_{\chc[A]{1,2}}}, given the configuration \{$A.2$\}.
%
%% Synchronization
%
Choices that share dimensions are subject to \textit{synchronization} under
selection. For example the expression:

\[ex_{\chc[A]{1,2}} = \chc[A]{b \Leftrightarrow c, a} \wedge d \wedge \chc[A]{\neg c,
    \chc[B]{\bot, \top}} \]

has two choices ($A$, and $B$) and four variants (two variants for two choices).
Given the configuration \{$A.1$\} the resultant variant would then be:

\[ \tsel[A.1]{ex_{\chc[A]{1,2}}}= b \Leftrightarrow c \wedge d \wedge \neg c \]

Notice that selection occurred for both choices indexed by $A$, which removed a
nested choice indexed by $B$. Indeed, if we were to select the second
alternative on $A$ then the resultant would still be variational instead of
plain.
